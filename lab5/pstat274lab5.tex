% Options for packages loaded elsewhere
\PassOptionsToPackage{unicode}{hyperref}
\PassOptionsToPackage{hyphens}{url}
%
\documentclass[
]{article}
\usepackage{amsmath,amssymb}
\usepackage{iftex}
\ifPDFTeX
  \usepackage[T1]{fontenc}
  \usepackage[utf8]{inputenc}
  \usepackage{textcomp} % provide euro and other symbols
\else % if luatex or xetex
  \usepackage{unicode-math} % this also loads fontspec
  \defaultfontfeatures{Scale=MatchLowercase}
  \defaultfontfeatures[\rmfamily]{Ligatures=TeX,Scale=1}
\fi
\usepackage{lmodern}
\ifPDFTeX\else
  % xetex/luatex font selection
\fi
% Use upquote if available, for straight quotes in verbatim environments
\IfFileExists{upquote.sty}{\usepackage{upquote}}{}
\IfFileExists{microtype.sty}{% use microtype if available
  \usepackage[]{microtype}
  \UseMicrotypeSet[protrusion]{basicmath} % disable protrusion for tt fonts
}{}
\makeatletter
\@ifundefined{KOMAClassName}{% if non-KOMA class
  \IfFileExists{parskip.sty}{%
    \usepackage{parskip}
  }{% else
    \setlength{\parindent}{0pt}
    \setlength{\parskip}{6pt plus 2pt minus 1pt}}
}{% if KOMA class
  \KOMAoptions{parskip=half}}
\makeatother
\usepackage{xcolor}
\usepackage[margin=1in]{geometry}
\usepackage{color}
\usepackage{fancyvrb}
\newcommand{\VerbBar}{|}
\newcommand{\VERB}{\Verb[commandchars=\\\{\}]}
\DefineVerbatimEnvironment{Highlighting}{Verbatim}{commandchars=\\\{\}}
% Add ',fontsize=\small' for more characters per line
\usepackage{framed}
\definecolor{shadecolor}{RGB}{248,248,248}
\newenvironment{Shaded}{\begin{snugshade}}{\end{snugshade}}
\newcommand{\AlertTok}[1]{\textcolor[rgb]{0.94,0.16,0.16}{#1}}
\newcommand{\AnnotationTok}[1]{\textcolor[rgb]{0.56,0.35,0.01}{\textbf{\textit{#1}}}}
\newcommand{\AttributeTok}[1]{\textcolor[rgb]{0.13,0.29,0.53}{#1}}
\newcommand{\BaseNTok}[1]{\textcolor[rgb]{0.00,0.00,0.81}{#1}}
\newcommand{\BuiltInTok}[1]{#1}
\newcommand{\CharTok}[1]{\textcolor[rgb]{0.31,0.60,0.02}{#1}}
\newcommand{\CommentTok}[1]{\textcolor[rgb]{0.56,0.35,0.01}{\textit{#1}}}
\newcommand{\CommentVarTok}[1]{\textcolor[rgb]{0.56,0.35,0.01}{\textbf{\textit{#1}}}}
\newcommand{\ConstantTok}[1]{\textcolor[rgb]{0.56,0.35,0.01}{#1}}
\newcommand{\ControlFlowTok}[1]{\textcolor[rgb]{0.13,0.29,0.53}{\textbf{#1}}}
\newcommand{\DataTypeTok}[1]{\textcolor[rgb]{0.13,0.29,0.53}{#1}}
\newcommand{\DecValTok}[1]{\textcolor[rgb]{0.00,0.00,0.81}{#1}}
\newcommand{\DocumentationTok}[1]{\textcolor[rgb]{0.56,0.35,0.01}{\textbf{\textit{#1}}}}
\newcommand{\ErrorTok}[1]{\textcolor[rgb]{0.64,0.00,0.00}{\textbf{#1}}}
\newcommand{\ExtensionTok}[1]{#1}
\newcommand{\FloatTok}[1]{\textcolor[rgb]{0.00,0.00,0.81}{#1}}
\newcommand{\FunctionTok}[1]{\textcolor[rgb]{0.13,0.29,0.53}{\textbf{#1}}}
\newcommand{\ImportTok}[1]{#1}
\newcommand{\InformationTok}[1]{\textcolor[rgb]{0.56,0.35,0.01}{\textbf{\textit{#1}}}}
\newcommand{\KeywordTok}[1]{\textcolor[rgb]{0.13,0.29,0.53}{\textbf{#1}}}
\newcommand{\NormalTok}[1]{#1}
\newcommand{\OperatorTok}[1]{\textcolor[rgb]{0.81,0.36,0.00}{\textbf{#1}}}
\newcommand{\OtherTok}[1]{\textcolor[rgb]{0.56,0.35,0.01}{#1}}
\newcommand{\PreprocessorTok}[1]{\textcolor[rgb]{0.56,0.35,0.01}{\textit{#1}}}
\newcommand{\RegionMarkerTok}[1]{#1}
\newcommand{\SpecialCharTok}[1]{\textcolor[rgb]{0.81,0.36,0.00}{\textbf{#1}}}
\newcommand{\SpecialStringTok}[1]{\textcolor[rgb]{0.31,0.60,0.02}{#1}}
\newcommand{\StringTok}[1]{\textcolor[rgb]{0.31,0.60,0.02}{#1}}
\newcommand{\VariableTok}[1]{\textcolor[rgb]{0.00,0.00,0.00}{#1}}
\newcommand{\VerbatimStringTok}[1]{\textcolor[rgb]{0.31,0.60,0.02}{#1}}
\newcommand{\WarningTok}[1]{\textcolor[rgb]{0.56,0.35,0.01}{\textbf{\textit{#1}}}}
\usepackage{graphicx}
\makeatletter
\def\maxwidth{\ifdim\Gin@nat@width>\linewidth\linewidth\else\Gin@nat@width\fi}
\def\maxheight{\ifdim\Gin@nat@height>\textheight\textheight\else\Gin@nat@height\fi}
\makeatother
% Scale images if necessary, so that they will not overflow the page
% margins by default, and it is still possible to overwrite the defaults
% using explicit options in \includegraphics[width, height, ...]{}
\setkeys{Gin}{width=\maxwidth,height=\maxheight,keepaspectratio}
% Set default figure placement to htbp
\makeatletter
\def\fps@figure{htbp}
\makeatother
\setlength{\emergencystretch}{3em} % prevent overfull lines
\providecommand{\tightlist}{%
  \setlength{\itemsep}{0pt}\setlength{\parskip}{0pt}}
\setcounter{secnumdepth}{-\maxdimen} % remove section numbering
\ifLuaTeX
  \usepackage{selnolig}  % disable illegal ligatures
\fi
\IfFileExists{bookmark.sty}{\usepackage{bookmark}}{\usepackage{hyperref}}
\IfFileExists{xurl.sty}{\usepackage{xurl}}{} % add URL line breaks if available
\urlstyle{same}
\hypersetup{
  pdftitle={Psatat274\_lab5},
  hidelinks,
  pdfcreator={LaTeX via pandoc}}

\title{Psatat274\_lab5}
\author{}
\date{\vspace{-2.5em}2023-10-30}

\begin{document}
\maketitle

\begin{enumerate}
\def\labelenumi{\arabic{enumi}.}
\tightlist
\item
  We will analyze monthly milk production measured in pounds per from
  Jan.~1962 to Dec.~1975 from the package tsdl as Lab 4 (if you want to
  re-install tsdl, please refer to Lab 4). Let's denote the time series
  milk as Xt.
\end{enumerate}

\begin{Shaded}
\begin{Highlighting}[]
\FunctionTok{library}\NormalTok{(tsdl)}
\NormalTok{milk }\OtherTok{\textless{}{-}} \FunctionTok{subset}\NormalTok{(tsdl, }\DecValTok{12}\NormalTok{, }\StringTok{"Agriculture"}\NormalTok{)[[}\DecValTok{3}\NormalTok{]]}
\FunctionTok{plot}\NormalTok{(milk)}
\end{Highlighting}
\end{Shaded}

\includegraphics{pstat274lab5_files/figure-latex/unnamed-chunk-1-1.pdf}

\begin{enumerate}
\def\labelenumi{(\alph{enumi})}
\tightlist
\item
  Explain why the series milk looks not stationary. To make series milk
  stationary, please difference at lag 12 and then at lag 1.
\end{enumerate}

Series milk looks not stationary because the mean of series milk is not
constant over time, which means there is trend for milk series.

\begin{Shaded}
\begin{Highlighting}[]
\CommentTok{\# First difference at lag 12 }
\NormalTok{dmilk }\OtherTok{\textless{}{-}} \FunctionTok{diff}\NormalTok{(milk, }\AttributeTok{lag =} \DecValTok{12}\NormalTok{)}

\CommentTok{\# Then difference the result at lag 1 }
\NormalTok{ddmilk }\OtherTok{\textless{}{-}} \FunctionTok{diff}\NormalTok{(dmilk, }\AttributeTok{lag =} \DecValTok{1}\NormalTok{)}
\end{Highlighting}
\end{Shaded}

\begin{enumerate}
\def\labelenumi{(\alph{enumi})}
\setcounter{enumi}{1}
\tightlist
\item
\end{enumerate}

\begin{Shaded}
\begin{Highlighting}[]
\NormalTok{op }\OtherTok{\textless{}{-}} \FunctionTok{par}\NormalTok{(}\AttributeTok{mfrow =} \FunctionTok{c}\NormalTok{(}\DecValTok{1}\NormalTok{,}\DecValTok{2}\NormalTok{))}
\FunctionTok{acf}\NormalTok{(ddmilk, }\AttributeTok{lag.max =} \DecValTok{50}\NormalTok{, }\AttributeTok{main =} \StringTok{"ACF: First and Seasonally Differenced Time Series"}\NormalTok{)}
\FunctionTok{pacf}\NormalTok{(ddmilk, }\AttributeTok{lag.max =} \DecValTok{50}\NormalTok{, }\AttributeTok{main =} \StringTok{"PACF: First and Seasonally Differenced Time Series"}\NormalTok{)}
\end{Highlighting}
\end{Shaded}

\includegraphics{pstat274lab5_files/figure-latex/unnamed-chunk-3-1.pdf}

\begin{Shaded}
\begin{Highlighting}[]
\NormalTok{op }\OtherTok{\textless{}{-}} \FunctionTok{par}\NormalTok{(}\AttributeTok{mfrow =} \FunctionTok{c}\NormalTok{(}\DecValTok{1}\NormalTok{,}\DecValTok{2}\NormalTok{))}
\FunctionTok{acf}\NormalTok{(ddmilk, }\AttributeTok{lag.max =} \DecValTok{12}\NormalTok{, }\AttributeTok{main =} \StringTok{"ACF: First and Seasonally Differenced Time Series"}\NormalTok{)}
\FunctionTok{pacf}\NormalTok{(ddmilk, }\AttributeTok{lag.max =} \DecValTok{12}\NormalTok{, }\AttributeTok{main =} \StringTok{"PACF: First and Seasonally Differenced Time Series"}\NormalTok{)}
\end{Highlighting}
\end{Shaded}

\includegraphics{pstat274lab5_files/figure-latex/unnamed-chunk-4-1.pdf}

\begin{enumerate}
\def\labelenumi{(\alph{enumi})}
\setcounter{enumi}{2}
\tightlist
\item
\end{enumerate}

Modeling the seasonal part (P, D, Q): For this part, focus on the
seasonal lags h = 1s, 2s, etc. We applied one seasonal differencing so D
= 1 at lag s = 12. The ACF shows a strong peak at h = 1s. A good choice
for the MA part could be Q=1. The PACF shows one strong peaks at h = 1s.
also for 2s,4s. A good choice for the AR part could be P = 1, or 2,4.
Modeling the non-seasonal part (p , d, q): In this case focus on the
within season lags, h = 1,. . . ,11. We applied one differencing to
remove the trend: d = 1 The ACF and PACF seems to be tailing off A good
choice for the MA part could be q = 0 ,and a good choice for the AR part
could be p = 0.

two possible model SARIMA(0, 1, 0) × (2, 1, 1)12 or SARIMA(0, 1, 0) *
(4, 1, 1)12

\begin{enumerate}
\def\labelenumi{(\alph{enumi})}
\setcounter{enumi}{3}
\tightlist
\item
  For SARIMA(0, 1, 0) × (2, 1, 1)12
\end{enumerate}

\begin{Shaded}
\begin{Highlighting}[]
\CommentTok{\# install.packages("astsa")}
\FunctionTok{library}\NormalTok{(astsa)}
\NormalTok{fit.i }\OtherTok{\textless{}{-}} \FunctionTok{sarima}\NormalTok{(}\AttributeTok{xdata =}\NormalTok{ milk,}\AttributeTok{p =} \DecValTok{0}\NormalTok{, }\AttributeTok{d =} \DecValTok{1}\NormalTok{, }\AttributeTok{q =} \DecValTok{0}\NormalTok{,}
                 \AttributeTok{P =} \DecValTok{2}\NormalTok{ , }\AttributeTok{D =} \DecValTok{1}\NormalTok{, }\AttributeTok{Q =} \DecValTok{1}\NormalTok{, }\AttributeTok{S =} \DecValTok{12}\NormalTok{, }\AttributeTok{details =}\NormalTok{ F)}
\NormalTok{fit.i}\SpecialCharTok{$}\NormalTok{fit}
\end{Highlighting}
\end{Shaded}

\begin{verbatim}
## 
## Call:
## arima(x = xdata, order = c(p, d, q), seasonal = list(order = c(P, D, Q), period = S), 
##     include.mean = !no.constant, transform.pars = trans, fixed = fixed, optim.control = list(trace = trc, 
##         REPORT = 1, reltol = tol))
## 
## Coefficients:
##         sar1    sar2     sma1
##       0.0334  0.0196  -0.7017
## s.e.  0.1533  0.1209   0.1400
## 
## sigma^2 estimated as 34.44:  log likelihood = -459.64,  aic = 927.28
\end{verbatim}

Checking coefficients: From the above coefficients table,sar1 and sar2
are not significant because the confidence interval of the estimated
coefficient contains 0. Therefore, we should set these coefficients to
0.

\begin{Shaded}
\begin{Highlighting}[]
\NormalTok{fit.i1 }\OtherTok{\textless{}{-}} \FunctionTok{sarima}\NormalTok{(}\AttributeTok{xdata =}\NormalTok{ milk, }\AttributeTok{p =} \DecValTok{0}\NormalTok{, }\AttributeTok{d =} \DecValTok{1}\NormalTok{, }\AttributeTok{q =} \DecValTok{0}\NormalTok{,}
                 \AttributeTok{P =} \DecValTok{2}\NormalTok{ , }\AttributeTok{D =} \DecValTok{1}\NormalTok{, }\AttributeTok{Q =} \DecValTok{1}\NormalTok{, }\AttributeTok{S =} \DecValTok{12}\NormalTok{, }
                 \AttributeTok{fixed=}\FunctionTok{c}\NormalTok{(}\DecValTok{0}\NormalTok{, }\DecValTok{0}\NormalTok{, }\ConstantTok{NA}\NormalTok{),}
                 \AttributeTok{details =}\NormalTok{ F)}
\NormalTok{fit.i1}\SpecialCharTok{$}\NormalTok{fit}
\end{Highlighting}
\end{Shaded}

\begin{verbatim}
## 
## Call:
## arima(x = xdata, order = c(p, d, q), seasonal = list(order = c(P, D, Q), period = S), 
##     include.mean = !no.constant, transform.pars = trans, fixed = fixed, optim.control = list(trace = trc, 
##         REPORT = 1, reltol = tol))
## 
## Coefficients:
##       sar1  sar2     sma1
##          0     0  -0.6750
## s.e.     0     0   0.0752
## 
## sigma^2 estimated as 34.47:  log likelihood = -459.66,  aic = 923.33
\end{verbatim}

From the above output, we can write the model as:
\((1 − B)(1 − B^{12})Y_t = (1 − 0.6750B^{12})Zt\) So, it is a
SARMA(p=0,d=1,q=0)*(P =2,D=1,Q=1)s=12

Check the model stationarity/invertibility: Lastly, we check the model
stationarity/invertibility. SMA part:

\begin{Shaded}
\begin{Highlighting}[]
\CommentTok{\# plot.roots(NULL,polyroot(c(1,− 0.6750)), main="roots of SAR part")}
\end{Highlighting}
\end{Shaded}

I don't why i could not us plot.roots function. here the root is outside
the unit circle, representing this model is both stationary and
invertible.

For SARIMA(0, 1, 0) * (4, 1, 1)12

\begin{Shaded}
\begin{Highlighting}[]
\NormalTok{fit.i }\OtherTok{\textless{}{-}} \FunctionTok{sarima}\NormalTok{(}\AttributeTok{xdata =}\NormalTok{ milk,}\AttributeTok{p =} \DecValTok{0}\NormalTok{, }\AttributeTok{d =} \DecValTok{1}\NormalTok{, }\AttributeTok{q =} \DecValTok{0}\NormalTok{,}
                 \AttributeTok{P =}\DecValTok{4}\NormalTok{ , }\AttributeTok{D =} \DecValTok{1}\NormalTok{, }\AttributeTok{Q =} \DecValTok{1}\NormalTok{, }\AttributeTok{S =} \DecValTok{12}\NormalTok{, }\AttributeTok{details =}\NormalTok{ F)}
\NormalTok{fit.i}\SpecialCharTok{$}\NormalTok{fit}
\end{Highlighting}
\end{Shaded}

\begin{verbatim}
## 
## Call:
## arima(x = xdata, order = c(p, d, q), seasonal = list(order = c(P, D, Q), period = S), 
##     include.mean = !no.constant, transform.pars = trans, fixed = fixed, optim.control = list(trace = trc, 
##         REPORT = 1, reltol = tol))
## 
## Coefficients:
##          sar1     sar2     sar3     sar4    sma1
##       -0.6659  -0.4356  -0.3459  -0.4243  0.0197
## s.e.   0.2086   0.1511   0.1144   0.0925  0.2348
## 
## sigma^2 estimated as 30.36:  log likelihood = -454.12,  aic = 920.25
\end{verbatim}

Checking coefficients: From the above coefficients table,sma1 is not
significant because the confidence interval of the estimated coefficient
contains 0. Therefore, we should set these coefficients to 0.

\begin{Shaded}
\begin{Highlighting}[]
\NormalTok{fit.i1 }\OtherTok{\textless{}{-}} \FunctionTok{sarima}\NormalTok{(}\AttributeTok{xdata =}\NormalTok{ milk, }\AttributeTok{p =} \DecValTok{0}\NormalTok{, }\AttributeTok{d =} \DecValTok{1}\NormalTok{, }\AttributeTok{q =} \DecValTok{0}\NormalTok{,}
                 \AttributeTok{P =} \DecValTok{4}\NormalTok{ , }\AttributeTok{D =} \DecValTok{1}\NormalTok{, }\AttributeTok{Q =} \DecValTok{1}\NormalTok{, }\AttributeTok{S =} \DecValTok{12}\NormalTok{, }
                 \AttributeTok{fixed=}\FunctionTok{c}\NormalTok{(}\ConstantTok{NA}\NormalTok{,}\ConstantTok{NA}\NormalTok{,}\ConstantTok{NA}\NormalTok{,}\ConstantTok{NA}\NormalTok{,}\DecValTok{0}\NormalTok{),}
                 \AttributeTok{details =}\NormalTok{ F)}
\NormalTok{fit.i1}\SpecialCharTok{$}\NormalTok{fit}
\end{Highlighting}
\end{Shaded}

\begin{verbatim}
## 
## Call:
## arima(x = xdata, order = c(p, d, q), seasonal = list(order = c(P, D, Q), period = S), 
##     include.mean = !no.constant, transform.pars = trans, fixed = fixed, optim.control = list(trace = trc, 
##         REPORT = 1, reltol = tol))
## 
## Coefficients:
##          sar1     sar2     sar3     sar4  sma1
##       -0.6497  -0.4260  -0.3412  -0.4230     0
## s.e.   0.0790   0.0979   0.0993   0.0917     0
## 
## sigma^2 estimated as 30.36:  log likelihood = -454.13,  aic = 918.26
\end{verbatim}

here is the model
\((1-0.6497B^{12}-0.426B^{24}-0.3412B^{36}-0.423B^{12})X_{t}=Z_{t}\)

\begin{Shaded}
\begin{Highlighting}[]
\CommentTok{\# plot.roots(NULL,polyroot(c(1,{-}0.6497,{-}0.426,{-}0.3412,{-}0.4230)), main="roots of SAR part")}
\end{Highlighting}
\end{Shaded}

There exist roots inside the circle, indicating it is not stationary.

The final fitting model would be
\((1 − B)(1 − B^{12})Y_t = (1 − 0.6750B^{12})Zt\).

\end{document}
